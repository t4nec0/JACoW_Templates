\documentclass[acus]{JAC2003}

%%
%%  This file was updated in March 2011 by T. Satogata to be in line with Word templates.
%%
%%  Use \documentclass[boxit]{JAC2003}
%%  to draw a frame with the correct margins on the output.
%%
%%  Use \documentclass[acus]{JAC2003}
%%  for US letter paper layout
%%
%%  Use \documentclass[a4paper]{JAC2003}
%%  for A4 paper layout
%%

\usepackage{graphicx}
\usepackage{booktabs}

%%
%%   VARIABLE HEIGHT FOR THE TITLE BOX (default 35mm)
%%

\setlength{\titleblockheight}{27mm}

\begin{document}
\title{PREPARATION OF PAPERS FOR JACoW CONFERENCES\thanks{ Work supported by ...}}

\author{J. Poole, C. Petit-Jean-Genaz, CERN, Geneva, Switzerland\\
C.E. Eyberger\thanks{ cee@aps.anl.gov}, ANL, Argonne, IL 60439, USA}

\maketitle

\begin{abstract}
   Many conference series have adopted the same standards for electronic
   publication and have joined the Joint Accelerator Conference Website (JACoW) 
   collaboration~\cite{accelconf-ref} for the publication of their proceedings. 
   This document describes the common requirements for the submission of papers
   to these conferences. Please consult individual conference 
   information for page limits, method of electronic submission, etc.
   It is not intended that this should be a tutorial in word processing;
   the aim is to explain the particular requirements for electronic publication 
   at these conference series.
\end{abstract}

\section{SUBMISSION OF PAPERS}
Each author should submit the PostScript and all of the source files (text and figures), 
to enable the paper to be reconstructed if there are processing difficulties.

\section{MANUSCRIPTS}
Templates are provided for recommended software and authors are
advised to use them. Please consult the individual conference help pages if questions
arise.

\subsection{General Layout}

These instructions are a typical implementation of the
requirements. Manuscripts should have:
\begin{Itemize}
    \item  Either A4 (21.0\,cm~$\times$~29.7\,cm; 8.27\,in~$\times$~11.69\,in) or US
           letter size (21\,cm~$\times$~27.9\,cm; 8.5\,in~$\times$~11.0\,in) paper.
    \item  {\it Single-spaced} text in two columns of 82.5\,mm (3.25\,in) with 5.3\,mm
           (0.2\,in) separation. Newer versions of Word (2007, 2010) have a default spacing
           of 1.5 lines; authors must change this to 1 line.
    \item  The text located within the margins specified in Table~\ref{l2ea4-t1}
           to facilitate electronic processing of the PDF file.
\end{Itemize}
\begin{table}[hbt]
   \centering
   \caption{Margin Specifications}
   \begin{tabular}{lcc}
       \toprule
       \textbf{Margin} & \textbf{A4 Paper} & \textbf{US Letter Paper} \\ 
       \midrule
           Top         & 37\,mm (1.46\,in)            & 0.75\,in (19\,mm)        \\
          Bottom     & 19\,mm (0.75\,in)            & 0.75\,in (19\,mm)        \\
           Left         & 20\,mm (0.79\,in)            & 0.79\,in (20\,mm)        \\
           Right       & 20\,mm (0.79\,in)            & 1.02\,in (26\,mm)        \\
       \bottomrule
   \end{tabular}
   \label{l2ea4-t1}
\end{table}

The layout of the text on the page is illustrated in
Fig.~\ref{l2ea4-f1}. Note that the paper's title and the author list should be the width of the
full page.Tables and figures may span the whole 170\,mm page width,
if desired (see Fig.~\ref{l2ea4-f2}), but full-width figures should be placed at
either the top or bottom of a page to ensure a proper flow of the text (Word templates only).

\begin{figure}[htb]
   \centering
   \includegraphics*[width=65mm]{JACpic_mc.eps}
   \caption{Layout of papers.}
   \label{l2ea4-f1}
\end{figure}

\subsection{Fonts}

In order to produce good Adobe Acrobat PDF files, 
authors using a \LaTeX\ template are asked to use only Times (in roman (standard), bold or
italic) and symbols from  the standard PostScript set of
fonts. In Word use only Symbol and, depending on your platform, Times or Times New Roman fonts in standard, bold or italic form. 

\begin{figure*}[tb]
    \centering
    \includegraphics*[width=168mm]{JACpic2v2.eps}
    \caption{Example of a full-width figure showing the JACoW Team at their annual meeting in 2008. 
    The figure carries a multi-line caption which has to be justified, rather than centered.}
    \label{l2ea4-f2}
\end{figure*}

\subsection{Title and Author List}

The title should use 14\,pt bold uppercase letters and be centred on the page.
Individual letters may be lowercase to avoid misinterpretation (e.g., mW, MW).
To include a funding support statement, put an asterisk after the title and a
footnote at the bottom of the first column on page 1; in \LaTeX\ use
$\backslash$\texttt{thanks}.
  
The names of authors, their organisations/affiliations and mailing addresses 
should be grouped by affiliation and listed in 12\,pt upper and lowercase letters.
The name of the submitting or primary author should be first, followed by 
the co-authors, alphabetically by affiliation.


\subsection{Section Headings}

Section headings should not be numbered. They should
use  12\,pt  bold  uppercase  letters  and  be  centred  in  the
column. All section headings should appear directly above
the text -- there should never be a column break between a heading and the
following paragraph.

\subsection{Subsection Headings}

Subsection  headings  should  not  be  numbered.  They
should use 12\,pt italic letters and be left aligned in the
column. Subsection headings should appear
directly above the text -- there should never be a column break between a
heading and the following paragraph.

\vspace*{6pt} {\bf Third-level Heading} is a new style, but authors must
bold text themselves; it should be used sparingly. See Table 2 for its
style details.

\subsection{Paragraph Text}

Paragraphs should use 10\,pt font and be justified (touch each side) in
the column. The beginning of each paragraph should be indented
approximately 3\,mm (0.13\,in). The last line of a paragraph should not be
printed by itself at the beginning of a column nor should the first line of
a paragraph be printed by itself at the end of a column.

\subsection{Figures, Tables and Equations}

Place figures and tables as close to the place of their mention as
possible. Lettering in figures and tables should be large enough to
reproduce clearly. Use of non-approved fonts in figures often leads to
problems when the files are processed. \LaTeX\ users -- please be sure to use non bitmapped
versions of Computer Modern fonts in equations (type 1 PostScript fonts are
required and their use is described in the JACoW help pages~\cite{jacow-help}).

All figures and tables must be given sequential numbers (1, 2, 3, etc.) and
have a caption (10\,pt font) placed below the figure or above the table being described.
Captions that are one line should be centred in the column, while captions
that span more than one line should be justified. The \LaTeX\ template uses the 'booktabs' 
package to format the tables.

A simple way to introduce figures into a Word document is to place them inside a table which has no borders. This is done in Word as follows:
\begin{Itemize}
\item	Insert a continuous section break.
\item	Insert two empty lines (makes later editing easier).
\item	Insert another continuous section break.
\item	Click between the two section breaks and Format $\rightarrow$ columns $\rightarrow$ Single.
\item	Table $\rightarrow$ Insert single column, two row table.
\item	Paste the figure in the first row and adjust the size as appropriate.
\item	Paste/Type the caption in the second row and apply figure caption style.
\item	Table $\rightarrow$ Table properties $\rightarrow$ Borders and shading $\rightarrow$ None.
\item	Table $\rightarrow$ Table properties $\rightarrow$ Alignment $\rightarrow$ Center.
\item	Table $\rightarrow$ Table properties $\rightarrow$ Text wrapping $\rightarrow$ None.
\item	Remove the blank lines from in and around the table.
\item	If necessary play with the cell spacing and other parameters to improve appearance.
\end{Itemize}

If a displayed equation needs a number, place it flush with the right
margin of the column (see Eq.~\ref{eq:units}). The equation itseld should be centred, if possible.
Units should be written
using the roman (standard) font, not the italic font.

\begin{equation}\label{eq:units}
    C_B={q^3\over 3\epsilon_{0} mc}=3.54\,\hbox{$\mu$eV/T}
\end{equation}

\subsection{References}

All bibliographical and web references should be numbered and listed at the
end of the paper in a section called ``References''. When referring to a
reference in the text, place the corresponding reference number in square
brackets~\cite{exampl-ref}. A URL may be added as part of a reference, but 
its hyperlink should NOT be added.

\subsection{Footnotes}

Footnotes on the title and author lines may be used for acknowledgements,
affiliations and e-mail addresses. A nonnumeric sequence of characters (*,
\dag, \ddag, \S) should be used. All other notes should be included in the
references section and use the normal numeric sequencing.

Word users--do not use Word's footnote feature (\textbf{Insert, Footnote}) to insert
footnotes as this will create formatting problems. Instead, insert footnotes manually in a text box at the bottom of the first column with a line at the top of the text box to separate the footnotes from the rest of the paper's text.  The easiest way to do this is to copy the text box from the JACoW template and paste it into your own document.  These 'pseudo footnotes' in the text box should only appear at the bottom of the first column on the first page~\cite{exampl-ref2}.

\subsection{Acronyms}

Acronyms should be defined the first time they appear.

\section{STYLES}

Table~\ref{style-tab} summarizes the fonts and spacings used in the styles of
a JACoW template (these are implemented in the \LaTeX\ class file).

\section{PAGE NUMBERS}

\textbf{DO NOT have any page numbers}. They will be added 
when the final proceedings are produced.

\section{TEMPLATES}

Templates and examples can be retrieved through Web browsers like Firefox
and Internet Explorer by saving to disk. See your local documentation for
details of how to do this~\cite{exampl-ref3}.

Template documents for the recommended word processing software are
available from the JACoW Website and exist for
\LaTeX, Microsoft Word (Mac and PC) and OpenOffice for US letter and A4 paper sizes.

Use the correct templates for  your paper size and version of Word.
Do not transport Microsoft Word documents across platforms, e.g.
Mac~$\leftrightarrow$~PC. When saving a Word 2010 file (PC), be sure
to click `Embed fonts' in the Save options. Fonts are embedded by default when printing to Postscript or PDF on Mac OSX.

Please see the information and help files for authors on the JACoW.org web site
for instructions  on  how to install templates in your Microsoft templates folder.

\section{CHECKLIST FOR ELECTRONIC PUBLICATION}

\begin{Itemize}
    \item  Use only Times or Times New Roman (standard, bold or italic) and Symbol 
    			fonts for text -- 10\,pt minimum except References which can be 9\,pt or 10\,pt.
    \item  Figures should use Times or Times New Roman (standard, bold or italic) and Symbol fonts when possible~-- 6\,pt minimum.
    \item  Check that the PostScript file prints correctly.
    \item  Check that there are no page numbers.
    \item  Check that the margins on the printed version are within $\pm$1\,mm of the specification.
    \item  \LaTeX\ users can check their margins by invoking the
           \texttt{boxit} option.
\end{Itemize}

\vspace*{-20pt}
\begin{table}[h]
    \setlength\tabcolsep{4pt}
    \caption{Summary of Styles}
    \label{style-tab}
    \begin{tabular}{llcc}
        \toprule
        \textbf{Style} & \textbf{Font}               & \textbf{Space}  & \textbf{Space} \\
                       &                             & \textbf{Before} & \textbf{After} \\ 
        \midrule
         Heading 1,        & 14\,pt                      & 0\,pt           & 3\,pt  \\
          PaperTitle             & Upper case except for       &                 &      \\
                       & required lower case letters &                 &      \\   %corrected 080515 vrws requred 
                       & Bold                        &                 &      \\ 
         \midrule
          Author list  & 12\,pt                      & 9\,pt           & 12\,pt \\
                       & Upper and Lower case        &                 &      \\ 
         \midrule
         Heading 2, & 12\,pt                      & 9\,pt           & 3\,pt  \\
         Section       & Uppercase                   &                 &      \\
         Heading              & bold                        &                 &      \\ 
        \midrule
         Heading 3,   & 12\,pt                      & 6\,pt           & 3\,pt  \\
         Subsection       & Initial Caps                &                 &      \\
         Heading              & Italic                      &                 &      \\ 
        \midrule
         Third-level   & 10\,pt                      & 6\,pt           & 0\,pt  \\
         Heading       & Initial Caps                &                 &      \\
                             & Bold                      &                 &      \\ 
        \midrule
         Figure        & 10\,pt                      & 3\,pt           & 6\,pt  \\
         Captions      &                             &                 &      \\
        \midrule
         Table         & 10\,pt                      & 3\,pt           & 3\,pt  \\
         Captions      &                             &                 &      \\
        \midrule
         Equations     & 10\,pt base font            & 12\,pt          & 12\,pt \\
        \midrule
         References    & 10\,pt, justified with      & 0\,pt           & 0\,pt  \\
                       & 0.25'' hanging indent          &                 &      \\
        \bottomrule
    \end{tabular}
\end{table}

\section{ACKNOWLEDGMENT}
Any acknowledgment should be in a separate section directly preceding
the References section.

\begin{thebibliography}{9}   % Use for  1-9  references
%\begin{thebibliography}{99} % Use for 10-99 references

\bibitem{accelconf-ref}
C. Petit-Jean-Genaz and J. Poole, ``JACoW, A service to the Accelerator Community,''
EPAC'04, Lucerne, July 2004, THZCH03,  p.~249, \texttt{http://www.JACoW.org}

\bibitem{jacow-help} A. Name and D. Person, Phys. Rev. Lett. 25 (1997) 56.

\bibitem{exampl-ref}
A.N. Other, ``A Very Interesting Paper,'' EPAC'96, Sitges, June 1996, MOPCH31, p. 7984 (1996),
\texttt{http://www.JACoW.org}  \{no period after URL\}

\bibitem{exampl-ref2}
F.E.~Black et al., {\it This is a Very Interesting Book}, (New York: Knopf, 2007), 52.

\bibitem{exampl-ref3}
G.B.~Smith et al., ``Title of Paper,'' MOXAP07, these proceedings.
\end{thebibliography}

\end{document}
